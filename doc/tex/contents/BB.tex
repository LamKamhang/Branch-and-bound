\section{分枝定界法}
\begin{lstlisting}[language=python]
	class IPsolver:
    def __init__(self, c, Aub, bub, Aeq, beq, bounds, tol=1.0E-8):
        self.c = np.array(c)
        self.Aub = np.array(Aub) if Aub else None
        self.bub = np.array(bub) if bub else None
        self.Aeq = np.array(Aeq) if Aeq else None
        self.beq = np.array(beq) if beq else None
        self.bounds = np.array(bounds) if bounds else None
        self.tol = tol

        self.solution = np.zeros(self.c.shape)
        self.optimum = -np.inf
        self.isFoundSolution = False

        self.cur_sol = np.zeros(self.c.shape)
        self.cur_opt = -np.inf

        self.max_branch_num = 5
        self.cur_branch_num = 0

\end{lstlisting}


定义如上所示整数规划求解器类,以c, Aub, bub, Aeq, beq, bounds和tol为输入。默认求解最大化问
题,其中c表示目标函数中各变量的系数;Aub为小于等于约束$Aub*x<=bub$中的Aub矩阵,bub为其中右边的系数;Aeq为等于约束$Aeq*x=beq$中的Aeq矩阵,beq为其中右边的系数;bounds为各变量的上下界约束;tol为整数容忍度。
初始化中设置最大分支数为max\_branch\_num,并初始化当前的解cur\_sol和目标函数值cur\_opt。

\begin{lstlisting}[language=python]
    def core_solve(self, c, Aub, bub, Aeq, beq, bounds):
        sol, opt = None, None
        res = linprog(-c, A_ub=Aub, b_ub=bub, A_eq=Aeq, b_eq=beq, bounds=bounds)
\end{lstlisting}

核心求解函数,输入与类构造函数相同,首先将问题松弛为一般的线性规划问题,调用线性规划求解器求解。注意,这里的线性规划求解器是求解最小化问题的,故将c反号。

\begin{lstlisting}[language=python]
	if res.success:
		opt = -res.fun
		sol = res.x
		self.update_opt(sol, opt)
\end{lstlisting}

若该线性规划求解成功,则拿解和值更新当前的解和目标函数值;否则不用分支,直接退出。

\begin{lstlisting}[language=python]
	if self.needBranch(sol, opt) and self.cur_branch_num < self.max_branch_num:
		index = self.getFirstNotInt(sol)
		to_round = sol[index]
		len_c = len(self.c)
\end{lstlisting}

进一步判断是否需要分支(该函数将在之后详细讲解),并判断branch次数是否超过阈值,若需要分支则获得第一个非整数的变量的索引,和解中的对应变量的值,并继续进行下述操作;否则,直接退出。

\begin{lstlisting}[language=python]
	Con1 = np.zeros((len_c, ))
	Con2 = np.zeros((len_c, ))
	Con1[index] = -1.0
	Con2[index] = 1.0
	if Aub is None and bub is None:
		A1 = Con1.reshape(1, len_c)
		A2 = Con2.reshape(1, len_c)
		B1 = np.array([-math.ceil(to_round)])
		B2 = np.array([math.floor(to_round)])
	else:
		A1 = np.vstack([Aub, Con1])
		A2 = np.vstack([Aub, Con2])
		B1 = np.hstack([bub, -math.ceil(to_round)])
		B2 = np.hstack([bub, math.floor(to_round)])
\end{lstlisting}

上述代码段将当前非整数索引进行划分,即分别添加x\_index<=-math.ceil(to\_round)和x\_index>=math.floor(to\_round)的约束到原来的小于等于约束中,变成两个不同的分支。

\begin{lstlisting}[language=python]
	self.cur_branch_num += 1
	self.core_solve(c, A1, B1, Aeq, beq, bounds) # right branch
	self.core_solve(c, A2, B2, Aeq, beq, bounds) # left branch
\end{lstlisting}

将分好支的变量继续调用核心求解器函数,并将分支次数加1。因为每次求解时,我们都保存了整数解,故不需要多余的代码来处理之后的结果。当各个分支运行完毕之后求解结束。

\begin{lstlisting}[language=python]
    def allInteger(self, sol):
        tmp = np.array([abs(x-np.round(x)) for x in list(sol)])
        return all(tmp <= self.tol)
\end{lstlisting}

该函数判断解是否都是整数。

\begin{lstlisting}[language=python]
    def getFirstNotInt(self, sol):
        tmp = np.array([abs(x-np.round(x)) for x in list(sol)])
        l = tmp > self.tol
        for i in range(len(l)):
            if l[i]: return i
        return -1
\end{lstlisting}
该函数取得解中第一个非整数的索引号。

\begin{lstlisting}[language=python]
    def needBranch(self, sol, opt):
        if self.allInteger(sol): return False
        elif opt <= self.cur_opt: return False
        return True
\end{lstlisting}

该函数判断当前解和值的情况下,是否需要分支。并非都是整数或者,该解的目标函数值比当前函数值要大,则需要分支。

\begin{lstlisting}[language=python]
    def update_opt(self, sol, opt):
        if self.allInteger(sol) and self.cur_opt<opt:
            self.cur_opt = opt
            self.cur_sol = sol.copy()
            self.isFoundSolution = True
\end{lstlisting}
该函数用于更新当前解和目标函数值,只有在解都是整数,并且当前函数值比该解的函数值小时,才更新。

\begin{lstlisting}[language=python]
    def solve(self):
        self.core_solve(self.c, self.Aub, self.bub, self.Aeq, self.beq, self.bounds)
        if self.isFoundSolution:
            self.solution = np.array([int(np.round(x)) for x in list(self.cur_sol)])
            self.optimum  = self.cur_opt
            return True
        else:
            return False
\end{lstlisting}
该函数是对核心求解器函数的一个封装,用于将构造求解器类的各个参数,作为输入传入core\_solve函数中求解,最后并将结果保存到最后结果self.solution和self.optimum中。