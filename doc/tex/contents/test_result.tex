\chapter{测试结果}

我们的测试数据除了使用助教在\url{http://www.cs.zju.edu.cn/algo/teaching/2018/zgc\_2018.html}提供的中等规模及大规模测试外,另外根据需求构造了几组测试样例,我们称之为小型测试样例。以此来检验我们程序的运行情况。下表中详细描述了我们的测试样例和测试情况:


\begin{table}[h]
	\caption{小型测试样例}
	\centering
	\begin{tabular}{|l|l|l|}
		\hline
		测试样例 & 测试目的 & 测试结果 \\
		\hline
		sample1 & 测试极小化优化问题 & pass \\
		\hline
		sample2 & 所有约束都是小于等于的极大化问题 & pass \\
		\hline
		sample3 & 常规例子 & pass \\
		\hline
		sample4 & 约束包含大于等于,小于等于及等于 & pass \\
		\hline
		sample5 & 测试无解的样例 & pass \\
		\hline
		sample6 & 有无穷解的样例 & pass \\
		\hline
		sample7 & 测试bland法则 & pass \\
		\hline
	\end{tabular}
\end{table}

经过了小型的测试以后,我们对助教提供的例子进行测试。下面是测试结果:

\begin{table}[h]
	\caption{大中规模测试}
	\centering
	\begin{tabular}{|l|l|l|}
		\hline
		测试样例 & 测试目的 & 测试结果 \\
		\hline
		case1 & 大规模测试1 & pending \\
		\hline
		case2 & 大规模测试2 & pending \\
		\hline
		case3 & 中等规模测试1 & pass \\
		\hline
		case4 & 中等规模测试2 & pass \\
		\hline
		case5 & 中等规模测试3 & pass \\
		\hline
	\end{tabular}
\end{table}



非常不幸的是,我们的程序无法接受大规模的测试,这可能归咎于我们的优化上没有做好,无法承受三千个变量的压力。这在后面分析的章节中,会稍微详细一点描述这个情况。