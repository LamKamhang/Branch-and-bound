\section{进程管理}

进程模块总的可以分为进程创建与结束、进程调度、进程等待机制、杀死进程等四个方面的功能。

进程创建用于在系统中创建新的进程,而新创建的进程会自动设定为当前进程子进程。进程结束为在进程结束之后将退出系统时进行的操作。可支持按照指定优先级创建新进程。

进程调度用于对系统中所同时存在的多个进程进行调度,合理分配硬件资源。

进程等待机制用于协调进程之间的同步,通过进程等待机制可以使得将一个进程挂起,而等到另一个指定进程完成之后才将挂起进程恢复执行。但在此机制中挂起进程与被等待进程之间必须满足父子关系。

杀死进程使得用户或者程序可以强制结束特定进程的执行,以灵活管理进程。

其中提供接口供用户使用的指令有:


\begin{table}[H]
  \centering
  \caption{进程模块提供的用户指令}
  \includegraphics[scale=0.8]{process/img/用户指令.pdf}
\end{table}

