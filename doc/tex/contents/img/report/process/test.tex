\section{进程管理: 多级反馈队列}

\subsection{进程创建与退出测试}

\texttt{execk 进程名}

为了测试进程创建功能,我们提供了\texttt{execk 进程名}测试指令。该指令的执行会在系统中新建一个进程,并且进程名即为输入指令后面的参数。测试结果照片如下。

\begin{figure}[H]
  \centering
  \includegraphics[scale=1]{process/img/1.png}
  \caption{进程创建、退出测试}
\end{figure}


在上图中,我们可以看到在我们输入\texttt{execk test}命令后,第一个模块起始为 \texttt{exec\_from\_kernel():begin}和\texttt{exec\_from\_kernel():end} ,在这之中完成了进程的创建,我们可以看到\texttt{execk return with 0}的返回值,这表明进程成功创建了。第二个模块起始为\texttt{runprog()},进入了内核进程的执行函数,我们可以看到\texttt{excute task pid = 2}说明进程现在正在执行,此后,我们可以看到一系列有关\texttt{pc\_exited()}的操作,说明当前内核进程已经执行完毕,正在退出。在最后我设置打印出了所有进程链表中的信息,我们可以看到存在一个状态为\texttt{TASK\_EXITED}的进程,表示进程已退出完毕。(在打印操作后会将进程移出所有进程链表)。



\texttt{execprio 进程优先级}

 

\begin{figure}[H]
  \centering
  \includegraphics[scale=1]{process/img/2.png}
  \caption{进程创建、退出测试}
\end{figure}


为了测试进程创建功能,我们提供了\texttt{execprio 进程优先级}测试指令。该指令的执行会在系统中新建一个进程,并且进程名即为输入指令后面的参数。测试结果照片如上。我们可以看到在我们输入\texttt{execprio 20}命令后,第一个模块以结束\texttt{exec\_from\_kernel():end} ,在这之中完成了进程的创建。第二个模块起始为\texttt{runprog()},进入了内核进程的执行函数,我们可以看到\texttt{current\_task: 2}的多次输出说明进程现在正在执行,此后,我们可以看到一系列有关\texttt{pc\_exited()}的操作,说明当前内核进程已经执行完毕,正在退出。(这是第一个版本的图片,因而打印输出和前图有所不同)

\subsection{进程调度测试}

为了测试多级就绪队列,看出它的动态更新优先级时间片,我们连续创建两个优先级相同的进程,查看测试结果。测试结果图片如下。

 
\begin{figure}[H]
  \centering
  \includegraphics[scale=1]{process/img/3.png}
  \caption{进程调度测试}
\end{figure}


首先,我们使用\texttt{execprio 4}两次,以创建两个name为\texttt{Process4}的进程。在图中第一部分是第二次优先级为4的进程执行结束,我们立即输入ps来打印所有进程链表信息,可以看到两个初始优先级都为4的进程都已完成执行,并退出。但是两个进程现在呈现的优先级已经不相同,并且具有不同的时间片。说明在调度执行的过程中,优先级和时间片确在动态更新。

\subsection{kill进程测试}

为了测试进程创建功能,我们提供了\texttt{kill 进程pid}测试指令。该指令的执行会在系统中新建一个进程,并且进程名即为输入指令后面的参数。测试结果照片如下。

 

\begin{figure}[H]
  \centering
  \includegraphics[scale=1]{process/img/4.png}
  \caption{kill进程测试}
\end{figure}

首先我们输入ps指令,以确保pid = 3的进程仍然在就绪队列中。如上图所示,此时该进程确实仍存在于系统之中,且处于就绪状态。然后我们输入\texttt{kill 3}指令,之后再次输入ps指令,从上图中可以看到,此时该进程已经不再存在于系统之中,这表明该进程已经成功被kill掉了。



\section{进程管理:UC/OS II}

\subsection{进程创建与退出测试}

\texttt{execk 进程名}

\begin{figure}[H]
  \centering
  \includegraphics[scale=0.6]{process/img/5.png}
  \caption{进程创建测试}
\end{figure}

在初始化后,我们可以使用\texttt{ps}命令,展示当前所有链表信息和任务就绪表的全部内容。我们可以看到仅IDLE进程(优先级数值为63)在任务就绪表中为\texttt{TASK\_TURE}。而其他优先级都为\texttt{TASK\_FALSE}。

\subsection{进程调度}

\begin{figure}[H]
  \centering
  \includegraphics[scale=0.6]{process/img/6.png}
  \caption{进程调度测试}
  \label{process-img-6}
\end{figure}


首先,我们使用了\texttt{execk test}创建了一个进程,在系统中分配了一个空闲的优先级53给新进程。进程创建后,我们立即打印了当前所有链表信息,和任务就绪表的全部内容。可以看到处于就绪表中的进程53和63在表中的状态是\texttt{TASK\_TURE},如图\ref{process-img-6}。

\begin{figure}[H]
  \centering
  \includegraphics[scale=0.6]{process/img/7.png}
  \caption{进程调度测试}
  \label{process-img-7}
\end{figure}

此时进程的优先级数值上高于kernel进程,但实际优先级意义上低于kernel进程,当kernel进程的时间片用完时,该进程被立即执行,再一次打印输出任务就绪表的全部内容,则显示该进程已不在就绪表中,如图\ref{process-img-7}。

