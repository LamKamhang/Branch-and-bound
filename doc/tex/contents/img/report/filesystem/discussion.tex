
实现一个操作系统工作量非常大,用一个学期来完成这个项目虽然足够,但依照学生的惯性一般都要拖到最後才写,我也不例外,我是到了验收前一个月才开始大部分工作,尽管开学初前几周助教就已经在周五下午的课讲解完FAT32与EXT2的实现。

实现的过程其实并没有遇到很多问题,主要花的时间都在PC上调试或上网查资料,也可能是因爲我做的是文件系统,开始开发之前我第一个实现的功能是PC上的模拟器,如此一来就不需要经常去实验室调试,非常方便(事实上我只去了3次,验证的结果跟在PC上模拟完全相同)。

然而我的队友以及其他同学们似乎不是如此,他们到最後一周工作量非常大丶看他们好像很痛苦的样子,虽然没有控制好进度是我们的问题,但如果老师能用其他方式更精确的检查进度应该会好一些。

原本我并不知道我选的操作系统班是教改班,然而当我知道期末要做这个作业时已经没办法换了(因爲其他课时间都跟着排好),因此只好努力完成要求,虽然我自己觉得我做的工作没有很突出,但经由自己实现操作系统的过程确实使我对於操作系统原理有非常深刻的理解,在复习期末考试的期间,看到课件上的内容都可以联想到内核中的代码,我觉得这是我在这个班最大的收获。毕业後大部分课上讲过的内容可能最终都会忘记,然而自己做过的操作系统会一直留在我的记忆中。

~\hfill 陈翰逸

