我实现的是内存部分。一开始看助教的代码,觉得内存部分已经实现的比较好了。尤其是虚拟内存部分,我觉得我在助教的基础上拓展一下,发挥一下,站在巨人的肩膀上总是好行事的。

所以把目标定在了页置换算法和内存池上。写好内存池之后才细看助教的代码,然后发现助教的TLB是有问题的。问题具体体现在无法真正映射虚拟内存。那内存池和页置换之前就要修复助教的BUG。

在修复助教的BUG的时候,发现一个底层的BUG,当系统处理异常的时候,他有时会跳转到\texttt{TLB\_refilling}异常。我觉得是硬件底层的问题。询问了另一个硬件小组的同学,他也这么觉得。

事情变得神秘了起来。我就把目标定在了使用红黑树管理\texttt{mm\_struct}中的\texttt{vma\_struct}和绕过TLB,写一个直接对vma层面操作的malloc和free函数。之所以写malloc/free也是为了验收的时候可以展示。同时这个malloc/free也给了之后版本的可拓展性。

说到可拓展性,VMA里面的合并接口,我都留了出来,假如之后可以再写页置换的话,还可以再写一写。

其他方面,比如像时间安排啥的。我都没什么问题,慢慢写的,也不紧张。但是没有地方可以调试是一个问题。实验室总是不开门,写到一定程度是一定要去实验室测试代码的。所以还是希望之后实验室能多开门。

最后,这个操作系统还是极大的提高了我的视野,我才知道硬件小组的人都这么强。可能只有强者才能学硬件。

~\hfill 李仁钟
