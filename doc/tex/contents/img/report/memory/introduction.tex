\section{内存管理}

本操作系统的内存管理结构设计包括物理内存的管理、用户进程的虚拟地址段管理和相应的申请内存系统调用。

对比助教组,此操作系统重构了slab,设计了进程下的红黑树管理VMA,完成了malloc和free设计。

对于物理内存,其由bootmem, buddy和slab三个子模块组成。直接与进程部分和虚拟内存交互的是slab模块。其通过缓存的方式提高申请以及释放内存的速度。所以对于助教的代码进行了重构,优化了代码结构。便于在其他部分,尤其是VMA的实现中调用。

对于VMA管理,使用了红黑树的结构,优化了使用速度,使得LRU算法可以在此基础上拓展。

为了配合VMA的实现,加入了对于用户虚拟内存的申请和释放函数。用户程序在执行的过程中可以利用malloc/free来申请和释放内存。

提供的用户指令:

\begin{table}[H]
  \centering
  \caption{用户指令}
  \begin{tabular}{|c|c|}
    \hline
    \texttt{slabone} & \makecell{申请1000字节的内存块,不释放。\\输出申请前后的Buddy信息} \\
    \hline
    \texttt{slabck} & \makecell{申请99个100字节的内存块,释放。\\输出申请和释放前后的Buddy信息} \\
    \hline
    \texttt{memc} & \makecell{malloc 10个10大小的虚拟内存,然后逐个free。\\输出VMA在\texttt{mm\_struct}的红黑树中的颜色} \\
    \hline
    \texttt{mem\_alltree} & \makecell{malloc 7个10大小的虚拟内存,然后从小地址到大地址free。\\输出每步中红黑树的BFS下的结构。} \\
    \hline
  \end{tabular}
\end{table}
