\section{Parse}
程序接受的输入格式是\textbf{.lp}文件,这种格式并没有办法很容易地给单纯形法作爲输入,因此需要一个转换工具。

首先需要定义一些类表示输入信息

异常,如果发先未知的语句会抛出异常,以及原因。


\begin{lstlisting}[language=C++]
    class ParseException
\end{lstlisting}

变量的上下界

\begin{lstlisting}[language=C++]
    class Bounds {
        public:
            // lower <= x && x <= upper
            int upper, lower;
    };
\end{lstlisting}

在约束式中的变量,有变量的索引、係数

\begin{lstlisting}[language=C++]
    class Variable {
        public:
            int coefficient;
            size_t index;
    };
\end{lstlisting}


约束,可以分$\leq, \geq, =$三种情况,然后包含一个容器装约束中的变量,还有一个常数。
左右顺序是: $C_1, C_2, ..., C_N \leq \text{Constant}$

\begin{lstlisting}[language=C++]
    class Condition {
        public:
            enum Type { eq, leq, geq };
            Type type;
            std::vector<Variable> variables;
            int constant;
    };
\end{lstlisting}


\textbf{Data}包含所有的数据:
\begin{itemize}
    \item 约束: \textbf{conditions}
    \item 变量的上下界: \textbf{bounds}
    \item 变量的索引: \textbf{indices}
    \item 目标函数: \textbf{function}
\end{itemize}

\begin{lstlisting}[language=C++]
    class Data {
        private:
            std::vector<Condition> conditions;
            std::vector< std::pair<size_t, Bounds> > bounds;
            std::vector<size_t> indices;
            std::vector<Variable> function;
    };
\end{lstlisting}


虽然输入类似\textbf{C}的风格,是个上下文无关文法,但是爲了方便就简化为正则语言(应该不会出现非正则的情况)。

\subsection{预处理}
\begin{itemize}
	\item 移除注释 $/\backslash \backslash *(.|\backslash n)*?\backslash \backslash */$
	\item 移除多馀空白$\^ \backslash \backslash s*\$$
	\item 移除换行,用空白取代$[\backslash n\backslash r]$
\end{itemize}


预处理之后,整个输入就可以当作一行,然后用\textbf{;}符号当作真正的换行重新分行,一行一行处理

对于每一行,可以分成几种类型
\begin{itemize}
    \item \textbf{int} 定义变量
    \item \textbf{max}, \textbf{min}
    \item 约束
\end{itemize}

\subsection{优化}

对于约束如果变量只有1个的情况,可以当作该变量的上下界,因爲我们实现的是分支定界法,所以这些信息可以对算法效率有帮助。

值得注意的事情是如果变量的係数是负数,大于、小于要交换。

\subsection{工具函数}

\begin{lstlisting}[language=C++]
    std::vector<std::string> Data::Split(const std::string & input, char delim);
    std::string Data::Join(const std::vector<std::string> & input);
\end{lstlisting}

\textbf{a;b; c}或\textbf{a, b,$\backslash$nc,d}这类代表多个元素合在一个字符串上的形式,因爲比较复杂,需要合并、分割这两种功能来实现分开。