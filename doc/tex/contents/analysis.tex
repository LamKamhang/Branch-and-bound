\chapter{分析与评价}
\section{时间复杂度}
\subsection{单纯形法}
如果采用了\textbf{Bland}法则选择非基变量进行转轴变换,我们时能够证明单纯形法在有限步内时一定能够终止的。单纯形法在实践中非常有效,并且比Fourier–Motzkin消去法\cite{Fourier}等早期方法有了很大的改进。 然而,在1972年,Klee和Minty\cite{EXP}给出了一个例子,即Klee-Minty立方体,表明由Dantzig制定的单形方法的最坏情况复杂度是指数时间。

\subsection{分枝定界法}
求解整数规划的精确解是NP困难的,我们没有多项式时间复杂度的算法求解。分枝定界法中,我们可能需要遍历所有的枝,所以需要$O(2^n)$次计算线性规划。而我们是使用单纯形法进行计算,所以这里我们的时间复杂度将是$O(2^n)\times O(2^n) \approx O(2^n)​$

\section{空间复杂度}
我们对于约束的存储是比较大开销的,使用的是密集的矩阵存储方式,即并没有使用稀疏矩阵,这在空间中是有极大的浪费,仅存储这个约束矩阵就需要$O(n^{2})$的空间了。

单纯形法额外使用的空间除了约束矩阵之外,没有更多的空间开销了。

而对于分枝定界法,则需要生枝。在测试过程中,随着变量数目的增多,分枝定界法会有明显的空间开销,每一枝都有自己的一个矩阵,如果不考虑优化问题,我们需要$O(2^{n}) \times O(n^2) \approx O(2^n)$的空间。这个空间的开销是非常大的。

\section{评价}
我们实现的实现过程分为了三个模块,一个模块处理IO,一个模块负责单纯形,一个模块负责分枝定界的演化。而这里,IO的过程利用了正则匹配,对于将\textbf{lp}文件转换为我们需要的\textbf{txt}格式是需要一定的时间的。如果文件比较大,譬如助教提供的case1和case2,这里的IO需要的时间就是几秒钟。

不过我们可以只去衡量我们的分枝定界的话,可以只从\textbf{txt}中进行文件读入。去考量我们的分枝定界法的性能。

不过由于我们的单纯形法和分枝定界法的实现都是基本实现,没有考虑更多的优化,导致运算比较慢。实际上,我们可以将一些已经定下结果的变量筛去,从一定程度上减少$n$这个维度的开销。另外,在分枝定界中,挑去与$bound$比较接近的值进行分枝,这样可能会有一定的优化效果。